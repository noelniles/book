\chapter{Greatest Common Divisor}
How many times does a smaller measure go into a larger measure?
How many liters are in a gallon? How many meters in a kilometer?
How many $\pi$ in a day? How many radii in a circumference? These kinds of
questions were asked thousands of years ago and have motivated all of fundamental
mathematics.

\section{Definitions and Properties}
If $u$ and $v$ are two non-zero integers, the \textit{greatest common divisor} $\gcd(u, v)$ is the
greatest number that evenly divides both $u$ and $v$. Every number divides zero so if $u = v = 0$ 
the definition doesn't make sense instead by convention we say $\gcd(0, 0) = 0$/. From the definition
it follows

\begin{align}
	\gcd(u, v) &= \gcd(v, u) \\
	\gcd(u, v) &= \gcd(-v, u) \\
	\gcd(u, 0) &= \left| u \right|
\end{align}

From the above equations we can assume that $u$ and $v$ are are positive integers. By the
\textit{Fundamental Theorem of Arithmetic} every positive integers can be factored into the form
\begin{equation}
	u = 2^{u_{2}} 3^{u_{3}} 5^{u_{5}} 7^{u_{7}}\ldots = \prod_{p \, \text{prime}}{p^{u_{p}}}
\end{equation}

Related to the GCD function is the \textit{lowest common multiple (LCM)} function. The LCM function is
the one you learned in school when adding fractions. LCM looks very similar to GCD.
\begin{align}
	\gcd(u, v) &= \prod_{p \, \text{prime}}{p^{\min(u_{p},v_{p})}} \\
	\lcm(u, v) &= \prod_{p \, \text{prime}}{p^{\max(u_{p},v_{p})}}
\end{align}

Using the definition of the GCD and LCM functions we can derive several useful identities.

\begin{align}
	\gcd(u, v)w &= \gcd{uv, uw}, \quad &&if w \geq 0 \\
	\lcm(u, v)w &= \lcm(uw, vw), \quad &&if w \geq 0 \\
	u \cdot v   &= \gcd(u, v) \cdot \lcm(u, v) \quad &&if u, v \geq 0 \\
	\gcd(\lcm(u, v), \lcm(u, w)) &= \lcm(u, \gcd(v, w)) \\
	\lcm(\gcd(u, v), gcd(u, w)) &= \lcm(u, gcd(v, w))
\end{align}


If $gcd(u, v) = 1$ we say that $u$ is relatively prime or coprime to $v$. Coprime numbers and their properties
are very import in group theory and will be discussed more later. Coprime numbers also arise in the RSA
and other encryption schemes. Therefore it is important as a programmer to understand greatest Common
divisor and be able to implement it efficiently and correctly. In the following sections we will discuss
various ways to implement the greatest common divisor algorithm and discuss which methods are the
best in practice.

\section{Implementations of the Greatest Common Divisor Algorithm}
\subsection{Euclid's Algorithm}
Every textbook on number theory that I have ever read begins with a discussion of a GCD algorithm that
is attributed to Euclid. Many algorithm and discrete math classes also begin with Euclid's algorithm.
Euclid's algorithm is often used to introduce recursive functions to beginning programmers. The popularity
of this method is due to it's usefulness and age. Euclid's GCD method can be found in \textit{Elements}
Book 7, Propositions 1 and 2, (\textit{c.} 300 B.C) but he probably didn't invent it. It has survived
over 2200 years to the present day with suprisingly few improvements by modern algebra methods.

In 

\begin{proposition}[To find the greatest common measure of two positive numbers]
$ $\newline
The greatest common measure of two lengths can be found by repeatedly subtracting
the smaller of the two from the larger until the smaller divides the larger.
\end{proposition}

\begin{proof}
Let $A, C$ be two positive integers. We want to find the greatest common divisor of $A$ and $C$.
If $C \mid A$ we are done becasue $C \mid C$ and $C$ is the largest divisor of itself.
Therefore $C$ is the greatest divisor of itself and it divides $A$ making it the greatest common
divisor.

If $C \nmid A$ then continually substract the smaller from the larger until some number remains
that divides the previous one. This will eventually terminate on some number or unity. If the result is
unity then it will certainly divide the previous number and $A, C$ are coprime.

Let  E be the remainder of $A$ divided by $C$. And let $F$ be the remainder of $C$ divided by $E$. Since $F$
divides $E$ and $E$ divides $C - F$, $F$ also divides $C - F$ meaning $F$ divides itself and $F$ divides $C$.
And $C$ divides $A - E$; therefore $F$ also divides $A - E$, but it also divides $E$ therefore it also
divides $E$ therefor it divides $A$. Thus, $F$ is a common divisor of $A$ and $C$.

Now we must show that $F$ is the \textit{greatest} common divisor. Suppose there exists a number $G > F$ that
divides both $A$ and $C$. $G$ divides $C$ and $C$ divides $A - E$ and $G$ divides $A - E$, so $G$ also divides
the whole of $A$ so it divides the remainder $E$. But $E$ divides $C - F$ so $G$ divides $C - F$ and $G$ divides
the whole of $C$ so it divides the remainder $F$. This leads to a contradiction with a larger number dividing
a smaller number. Therefore $F$ is the largest number that will divide both $A$ and $C$.
\end{proof}

When I think of the Euclidean GCD algorithm I think of learning long division when I was a child. Remember, drawing
the frame over the dividend and then placing the divisor on the left. Then you would try to think how many times
you could multiply the left number (the divisor) until you got a remainder that was less then the divisor. This
process is called Euclidean Division.

The Euclidean GCD algorithm lends itself naturally to recursion. Below are two recursive versions of Euclid's GCD algorithm.
The second one is due to Dijkstra and uses substraction rather than the modulus operator which should be more efficient.
However these are presented only for reference. For serious work with large numbers we can't use recursion. I tested both 
recurisive versions on my machine. The second version crash with a segmentation fault because the recursion tree greatest
too deep.

\subsection{Recursive Euclidean Algorithm with $\mod()$}
The basis of this algorithm is the fact that
\[
0 < n \leq m \implies \gcd(m, n) =
\begin{cases}
	n \quad &\text{if n divides m without remainder} \\
	\gcd(n, remainder \frac{m}{n}) \quad &\text{otherwise}
\end{cases}
\]
Notice that in this version I replaced the subtraction mentioned used in Euclid's original with the modulus
operator. This is possible because division is just repeated subtraction and we don't ever use the quotient 
just the remainder. Below is the implementation in Go.

\begin{lstlisting}
// RGCD is the recurisive gcd algorithm. This shouldn't be used.
func RGCD(m int32, n int32) int32 {
	if n == 0 {
		return m
	}
	return RGCD(n, m%n)
}
\end{lstlisting}


\subsection{Recursive GCD with Subtraction Operator}
The next version uses subtraction which is a simpler operation, but it also requires more operations which makes the recursion
tree too deep for this implementation to work in the worst case.

\begin{lstlisting}
// This is a slightly better GCD function, but it still
// shouldn't be used because of the recursion.
func RGCD2(a int32, b int32) int32 {
	if a == b {
		return a
	} else if a > b {
		return RGCD2(a - b, b)
	} else {
		return RGCD2(a, b - a)
	}
}
\end{lstlisting}

\section{Imperitive GCD}
A more memory friendly version of the Euclidean Algorithm is below. Notice that it uses tempory variables $u, v, r, x$. These
are used so that we don't modify the original values. There is also some error checking code because this is the method
I would choose in practice. 

\begin{lstlisting}
func GCD(a int32, b int32) int32 {
	var u int32
	var v int32
	var t int32
	var x int32

	if a < 0 && a < -math.MaxInt32 {
		fmt.Println("GCD: integer overflow")
		a = -a
	}
	if b < 0 && b < -math.MaxInt32 {
		fmt.Println("GCD: integer overflow")
		b = -b
	}
	if b == 0 {
		x = a
	} else {
		u = a
		v = b
		for v != 0 {
			t = u % v
			u = v
			v = t
		}
		x = u
	}
	return x
}
\end{lstlisting}

There are quicker GCD algorithms.

\section{Extended Greatest Common Divisor}
Euclid's algorithm is great but we often want to know what the coeffiecents
as well. What I mean is, what if we want to know the $x$ and $y$ that satisfy
$ax + by = \gcd (a,b)$

\begin{lstlisting}

func XGCD(a int32, b int32) (int32, int32, int32) {
	var u, v, u0, v0, u1, v1, u2, v2, q, r int32
	var aneg, bneg int32

	if a < 0 {
		if a < -math.MaxInt32 {
			fmt.Println("XGCD: integer overflow")
		}
		a = -a
		aneg = 1
	}

	if b < 0 {
		if b < -math.MaxInt32 {
			fmt.Println("XGCD: integer overflow")
		}
		b = -b
		bneg = 1
	}

	u1 = 1
	v1 = 0
	u2 = 0
	v2 = 1
	u = a
	v = b

	for v != 0 {
		q = u / v
		r = u % v
		u = v
		v = r
		u0 = u2
		v0 = v2
		u2 = u1 - q*u2
		v2 = v1 - q*v2
		u1 = u0
		v1 = v0
	}
	if aneg != 0 {
		u1 = -u1
	}
	if bneg != 0 {
		v1 = -v1
	}
	return u, u1, v1
}
\end{lstlisting}
