\chapter{In the begining there was Euclid}

\section{Greatest Common Divisor}

How many times does a smaller measure go into a larger measure?
How many liters are in a gallon? How many meters in a kilometer?
How many $\pi$ in a day? How many radii in a circumference? These kinds of
questions were asked thousands of years ago and have motivated all of fundamental
mathematics.

It is just these sorts of questions that fueled the creation of Euclid's
elements centuries ago. And this is why every number theory book that I have
ever read begins with Euclid's Greatest Common Divisor algorithm. I have often
thought that it would be nicer to learn about sets and groups first (after
I had learned about sets and groups of course), but 
when talking about sets and groups the GCD plays an important role in describing
their structure and properties, so I will stick with tradtion and begin with
Euclid's GCD algorithm.

First let us state the GCD algorithm the way Euclid did. Euclid thought of 
numbers as measures or lengths. So, we will use the same terminology. A 
number therefore is a measure between two points. For example if A and B
are points AB is a number.

\begin{proposition}[To find the greatest common measure of two numbers]
$ $\newline
Let $AB$ and $CD$ be two numbers with $AB$ the less. We want to find the greatest
common measure of $AB$ and $CD$. Either $AB$ can measure $CD$ or it cannot. If
$AB$ measures $CD$ then it is the greatest common measure because it measures
itself and itself is the greatest magnitude that can measure itself.

If $AB$ does not measure $CD$ we can repeatedly subtract the lesser from 
the greater until the remainder does measure $CD$. The remainder will be either
some number or 1 (unity, numero uno). If the remainder is 1 then $AB$ and $CD$
are incommenurable (A.K.A. coprime, relatively prime, RSA cnadidates).

\end{proposition}

When I think of the Euclidean GCD algorithm I think of learning long division
when I was a child. 

In future chapters we will see how valuable the GCD algorithm is. 

\begin{lstlisting}
func GCD(a int32, b int32) int32 {
	var u int32
	var v int32
	var t int32
	var x int32

	if a < 0 && a < -math.MaxInt32 {
		fmt.Println("GCD: integer overflow")
		a = -a
	}
	if b < 0 && b < -math.MaxInt32 {
		fmt.Println("GCD: integer overflow")
		b = -b
	}
	if b == 0 {
		x = a
	} else {
		u = a
		v = b
		for v != 0 {
			t = u % v
			u = v
			v = t
		}
		x = u
	}
	return x
}
\end{lstlisting}

\section{Extended Greatest Common Divisor}

\begin{lstlisting}

func XGCD(a int32, b int32) (int32, int32, int32) {
	var u, v, u0, v0, u1, v1, u2, v2, q, r int32
	var aneg, bneg int32

	if a < 0 {
		if a < -math.MaxInt32 {
			fmt.Println("XGCD: integer overflow")
		}
		a = -a
		aneg = 1
	}

	if b < 0 {
		if b < -math.MaxInt32 {
			fmt.Println("XGCD: integer overflow")
		}
		b = -b
		bneg = 1
	}

	u1 = 1
	v1 = 0
	u2 = 0
	v2 = 1
	u = a
	v = b

	for v != 0 {
		q = u / v
		r = u % v
		u = v
		v = r
		u0 = u2
		v0 = v2
		u2 = u1 - q*u2
		v2 = v1 - q*v2
		u1 = u0
		v1 = v0
	}
	if aneg != 0 {
		u1 = -u1
	}
	if bneg != 0 {
		v1 = -v1
	}
	return u, u1, v1
}
\end{lstlisting}
